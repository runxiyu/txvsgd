\section{Introduction}

Notes, while our origial document had a lot of links (which we cna replicate here) bare in mind that entry level 3 is basic, and I mean basic, 

Entry level 3
level 1
level 2


\chapter{NCFE Entry 3 Essential Digital Skills}

This course is split into five manageable units:

\section{Unit 1: Using devices and handling information}

PCs running GNU \ Linux
Phones running a free OS
Using f-droid to install ethical applications 

\section{Unit 2: Finding and evaluating information}

Searx and duckduckgo


    Duckduckgo
    Searx - metasearch engine https://searx.me/
        List of searx instances
        Snopyta Searx https://search.snopyta.org/
        Disroot SearX https://search.disroot.org/

https://www.openstreetmap.org/


Evaluating info

Good information, should always come from a reputable source,  for example peer reviewed academic articles are a good example.   Sense about science has a lot of information on finding and evaluating information. 

\href{https://senseaboutscience.org/}{Sense about Science}

has good info on this

\section{Unit 3: Managing and storing information}

Local file system,  use of graphical file managers
Use of cloud (next cloud)

\href{https://www.nextcloud.com}{Nextcloud}



\section{Unit 4: Identifying and solving technical problems}

Checking all cables etc are attched properly
Look for messages
sounds e.g bios beep codes (more advanced)
What OS are you using

man / info pages  https://linux.die.net/man/
rfc documentation

uname -a output

\section{Unit 5: Developing digital skills}





\chapter{NCFE Level 1 Essentials Digital Skills}

The text below is taken from another website, it is there to help us with each section and shouldbe removed for copy right reasons once complete.

\section{Unit 1: Using devices and handling information}

This skills area covers a basic understanding of hardware, software, operating systems and commonly used applications. Students will develop fundamental digital skills by learning how to manage and store information, and identify and solve simple technical issues.

\textbf{Hardware}

\href{https://www.youtube.com/watch?v=HB4I2CgkcCo}{Inside a computer} needs invidious link

CPU 
Storage SSD / Metal hard disk, 
RAM (Memory)
Motherboard and other components

GNU Linux
Free vs nonfree software within a Linux system

\section{Unit 2: Creating and editing}

Students will first learn to create and edit documents before moving on to creating and editing other types of digital media, such as images, audio files and videos. Eventually, students will be taught to use applications to edit, enhance and format different types of information for a range of purposes and audiences.

\begin{description}
  \item[LibreOffice-writer] \mbox{}\\ LibreOffice Word processing.
  \item[LibreOffice Documentation] \mbox{}\\ https://books.libreoffice.org/en/.
  \item[LaTex - Beamer] \mbox{}\\ Presentation module for LaTeX.
  \item CryptPad: text, spreadsheet, presentation, cloud and more https://cryptpad.fr/
  \item EtherPad - Collaborative Editor https://etherpad.org/
\end{description}

  Typesetting

    LaTeX https://www.latex-project.org/
        LaTeX Templates https://www.latextemplates.com/
        Overleaf - collaborative editor https://www.overleaf.com/
        Overleaf / LaTeX documentation https://www.overleaf.com/learn
        LearnLatex
        tables generator
        detexify
        Texstudio and other tools for editing
        
        Note that Overleaf is a good editor,  easy to use, however free software advocates point out it is using non free platform for hosting  
        
        Note LaTeX may be too advanced for level 1 


\section{Unit 3: Communicating}

The communication skills area requires students to demonstrate an understanding of electronic communications, such as email and video calls. Students will gain awareness and be able to use digital communication for a range of contexts and audiences.

e-mail

\begin{description}
  \item[Thunderbird] \mbox{}\\ Thunderbird e-mail client.
  \item[Disroot] \mbox{}\\ https://www.disroot.org.
  \item protonmail
\end{description}

Messaging

Use Privacy Friendly & Free Software messaging services

    Matrix https://matrix.org/
    Telegram https://telegram.org/
    Signal https://signal.org/en/
    Jami https://jami.net/
    Internet Relay Chat http://www.ircbeginner.com/
    XMPP (Messaging protocol) https://xmpp.org/
    XMPP clients:
        Gajim https://gajim.org/
        Conversations https://conversations.im/
        Dino https://dino.im/


video conferencing (replacements for Zoom, MS teams and google meet)

Use Free/Libre & Privacy friendly alternative videoconferencing software

    Jitsi - https://jitsi.org/
    Jitsi servers:
        meet.jit.si https://meet.jit.si/
        Disroot https://calls.disroot.org/
    BigBlueButton
    BigBlueButton servers:
        FairCam https://bbb.faircam.net/
        Senfcall https://senfcall.de/en




decentralised social media

\begin{description}
  \item[Thunderbird] \mbox{}\\ Thunderbird e-mail client.
  \item[Disroot] \mbox{}\\ https://www.disroot.org.
\end{description}

Point out,ideally we should run own instances, learn the skills to do so


\subsection{e-mail}
\begin{description}
  \item[Thunderbird] \mbox{}\\ Thunderbird e-mail client.
  \item[Disroot] \mbox{}\\ https://www.disroot.org.
\end{description}


\section{Unit 4: Transacting}

This skills area involves the ability to complete and submit an online form, comply with digital verification checks, and purchase an item or service online. Eventually, students will learn to compare products against other available online options and manage their transactional account settings.

Examples can be



    GNU Taler https://taler.net/en/index.html
        GNU Taler launched at Bern University of Applied Sciences

    Liberapay (non profit), for recurring payments https://en.liberapay.com/

We can also use cryptocurrencies:

    Bitcoin https://bitcoin.org/
    Monero - anonymous and untraceable https://getmonero.org/



Sign up to services such as Disroot, this gives to access to the services that they offer (cloud, e-mail and much more)

Sign up to services such as hosting websites,  conference signups etc


\section{Unit 5: Being safe and responsible online}

The final skills area is to understand the importance of digital wellbeing, students will learn about being responsible online. This will involve privacy and data protection, conducting best practice online behaviour, backing up data, and understanding the psychological health risks of online activity.

Use privacy friendly tools, rather than tools that collect huge amounts of data

Be familar with the UK / EU GDPR


Idea

Being responsible should also include downloading GPL software and or Creative commons and respecting the license useage)

\chapter{TQUK  Level 2 IT User Skills}

\section{Unit 1: Using IT to increase productivity}
\section{Unit 2: IT software fundamentals}

\begin{description}
  \item[GNU Licenses] \mbox{}\\ https://www.gnu.org/licenses/licenses.html.
\end{description}





\section{Unit 3: IT security for users}

Passwords

apg - password generator

Password Managers
ssh keys for passwordless login
two factor authentication

\section{Unit 4: Presentation software}

\begin{description}
  \item[LibreOffice-impress] \mbox{}\\ LibreOffice Presentation Module.
  \item[LibreOffice Documentation] \mbox{}\\ https://books.libreoffice.org/en/.
  \item[LaTex - Beamer] \mbox{}\\ Presentation module for LaTeX.
  \item CryptPad: text, spreadsheet, presentation, cloud and more https://cryptpad.fr/
\end{description}

\section{Unit 5: Spreadsheet software}

\begin{description}
  \item[LibreOffice-calc] \mbox{}\\ LibreOffice Spreadsheet Module.
  \item[LibreOffice Documentation] \mbox{}\\ https://books.libreoffice.org/en/.
  \item EtherCalc - Collaborative Spreadsheet https://ethercalc.net/
\end{description}



\section{Links}

https://www.gatewayqualifications.org.uk/qualification-area/digital-qualifications/essential-digital-skills/


https://codeberg.org/DigitalSkills/DigitalSkills

\section{Further reading}






    LibreOffice https://www.libreoffice.org/
        LibreOffice Templates https://www.libreofficetemplates.net/
        LibreOffice basics https://personaljournal.ca/libreoffice/
        LibreOffice Documentation https://books.libreoffice.org/en/
    The Document foundation https://www.documentfoundation.org/
    Using Open Document Formats (ODF) in your organisation https://www.gov.uk/guidance/using-open-document-formats-odf-in-your-organisation


Collaborative editing

    EtherPad - Collaborative Editor https://etherpad.org/

    EtherCalc - Collaborative Spreadsheet https://ethercalc.net/
        Both the above come as part of Disroot, Cryptpad (below) also has collaborative features)

    CryptPad: text, spreadsheet, presentation, cloud and more https://cryptpad.fr/
    
    


Join the Fediverse, decentralised social media

    Fediverse https://fediverse.party/
    What is the Fediverse https://torresjrjr.com/archive/2020-07-20-what-is-the-fediverse
    Mastodon https://joinmastodon.org/
    Mastodon user guide https://docs.framasoft.org/en/mastodon/User-guide.html
    Lemmy: Reddit alternative, link aggregator https://join.lemmy.ml/
peertube - decentralised video sharing

        
        

Use Free/Libre alternative forum software

    Discourse
	PhpBB forum,  so sign up to forum.tuxiversity.org 


        

https://datadetoxkit.org/en/home/


  

