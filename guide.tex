\documentclass{extbook}

\usepackage{manfnt}
\usepackage{hyperref}

\title{Bruh}
\author{Paul Sutton (\texttt{zleap}), Andrew Yu, and Ron Nazarov}

\setlength\parindent{2.3\parindent}

\makeatletter
% Dangerous Bend - Please only use this when there are at least two lines of text in the dangerous paragraph!
\def\hang{\hangindent \parindent}
\def\d@nger{\medbreak\begingroup\clubpenalty=10000
  \def\par{\endgraf\endgroup\medbreak} \noindent\hang\hangafter=-2
    \hbox to0pt{\hskip-\hangindent\dbend\hfill}\normalsize}
\outer\def\danger{\d@nger}
\def\dd@nger{\medbreak\begingroup\clubpenalty=10000
  \def\par{\endgraf\endgroup\medbreak} \noindent\hang\hangafter=-2
    \hbox to0pt{\hskip-\hangindent\dbend\kern1pt\dbend\hfill}\normalsize}
\outer\def\ddanger{\dd@nger}
\def\enddanger{\endgraf\endgroup}

\newcommand\ph[1]{\texttt{\textit{#1}}}
\newcommand\libera[1]{\href{https://web.libera.chat/?channel=\##1}{\texttt{\##1} on \texttt{irc.libera.chat}}}
\newcommand\youtube[2]{\href{https://invidious.snopyta.org/watch?v=#1}{#2}}
\makeatother

\begin{document}
\maketitle

\frontmatter
\chapter{Copyright}

Copyright \copyright{}  2022  Paul Sutton.\\
Copyright \copyright{}  2022  Andrew Yu.\\
Copyright \copyright{}  2022  Ron Nazarov.\\
Permission is granted to copy, distribute and/or modify this document
under the terms of the GNU Free Documentation License, Version 1.3
or any later version published by the Free Software Foundation;
with no Invariant Sections, no Front-Cover Texts, and no Back-Cover Texts.
A copy of the license is included in the section entitled ``GNU
Free Documentation License'' (Appendix \ref{label_fdl}).

\chapter{Preface}

This project has developed from a series of posts on a personal blog site run by zleap. This is a refactoring of that content. This will initially focus on helping people learn the command line for the GNU / Linux operating system.

In turn this will hopefully complement the hosting guides produced by Andrew Yu. As most of those assume at least some knowledge of the command line. So it makes sense to be familiar with this. Learning basic shell scripting, will also help you interpret and modify scripts for your own needs.

This is a big subject area, I am going to start off with a series of posts to take this back to basics, starting off with learning about the command line in modern UNIX-like operating systems and associated commands using the \texttt{bash} shell. Then move on to scripting using Bash.

Although most commands and tips will work on all UNIX-like operating systems, some less common usages and options of these utilities may be absent on some systems. This series focuses on usage of the GNU core utilities with occasional commands from util-linux, as GNU/Linux systems are the most common Free Software setups. Users of other UNIX-like operating systems may consult their system manual for specifics into a command.

This guide is intended for beginners, as some sort of a semi-gentle introduction. However, people with experience already may also find this guide helpful, and as such this guide will cover some 'more advanced' topics. In order to make it possible for many types of readers to read this guide effectively, the  sign
$$\vbox{\hbox{\dbend}\vskip 11pt}$$
is put at the beginning of some paragraphs to indicate a "hard level". If you are new to GNU/Linux systems and it's your first few reads, it is advised to skip these parts. (Yes, this is the ``{dangerous bend}'' sign that Donald Knuth uses.)

\tableofcontents

\mainmatter
\chapter{Preparing the Environment}

$$\vcenter{\hbox{\dbend\kern1pt\dbend\kern1pt\dbend\kern1pt\dbend\kern1pt\dbend\kern1pt\dbend\kern1pt\dbend\kern1pt\dbend\kern1pt\dbend\kern1pt\dbend\kern1pt\dbend\kern1pt\dbend}\vskip 11pt}$$
This chapter needs to be rewritten.

Firstly, we need access to a GNU/Linux system, there are numerous options here:

\begin{itemize}
\item Install GNU/Linux on your computer;
\item Purchase (or otherwise obtain) computers that use GNU/Linux "by default" or has mainline support thereof, such as these single-board computers:
  \begin{itemize}
  \item \href{https://beagleboard.org}{BeagleBone boards}
  \item \href{https://www.raspberrypi.org/}{Raspberry Pi}
  \item \href{http://www.orangepi.org/}{Orange Pi}
  \item \href{https://www.pine64.org/devices/single-board-computers/rock64/}{ROCK64}
  \item \href{https://www.pine64.org/devices/single-board-computers/pine-a64-lts/}{PINE A64-LTS}
  \end{itemize}
  or these laptops:
  \begin{itemize}
  \item \href{https://www.pine64.org/pinebook/}{Pinebook}
  \item \href{https://www.pine64.org/pinebook-pro/}{Pinebook Pro}
  \item Anything listed on \href{https://ryf.fsf.org/categories/laptops}{the FSF's RYF page}
  \end{itemize}
\item Install GNU/Linux in a virtual machine;
  \begin{itemize}
  \item \href{https://www.qemu.org/}{QEMU}
  \end{itemize}
\item Use a public-access GNU/Linux system, such as \texttt{tilde.team};
\item Use a service from \href{https://www.vern.cc}{Vern.cc};
\item Use a browser-based emulator such as \texttt{https://bellard.org/jslinux/};
\item Use a VM from \libera{tuxiversity};
\item \textit{et cetera}.
\end{itemize}

\chapter{Getting support with this series}

You can get some help via IRC (Internet Relay Chat) -- myself (zleap), Ron (noisytoot), and AndrewYu (Andrew) all hang out on IRC in the channel \libera{tuxiversity}. There are several ways to connect to this via a client. It is probably easier, at least at first, to use a web based interface.

You could also join or help setup a local group such as

\begin{itemize}
\item \href{https://www.codeclub.org.uk}{CodeClub}
\item \href{https://coderdojo.com/}{CoderDojo}
\end{itemize}

The previous section also has links to communities that can also provide help and support.


\chapter{Shell Basics}

We will be mainly working in the \texttt{bash} shell in this series. GNU \texttt{bash} is the most common shell in modern GNU/Linux systems, which is the primary focus of this series.

A \textit{shell} is, in general, the program that you interact with your computer with. When used in a more specific context it refers to a command-line shell that users type commands to. The shell usually does some interpretation on the user input and passes these options launching a program, or they modify the shell's internal environment.

Many users, mostly novice ones, are afraid of command-line interfaces as the stereotypical UNIX command-line is very terse. In rare fields this is still true, but things are generally better than in the old days. For example, visual text editors are widespread among users of bitmap displays and terse teletype-oriented ones like \texttt{ed} are fading away.

Some may ask why the command-line is in consideration at all as we have graphical interfaces that are much more user-friendly. It boils down to the fact that graphical interfaces are usually harder to extend for more features and for specific uses, and even in these cases where you would extend a GUI, knowledge of programming is still necessary, which would be similar to shell scripting. In most cases simple scripts would be easier to write anyway; besides, using the shell day-to-day gives you confidence when you need to write batch operations---and believe me, it made my life much easier.

Now, go ahead and open up a shell. On graphical systems, look for a program called ``Terminal'' or similar. If you see something similar to ``\verb|user@host:~$|'', ``\verb|[user@host ~] $|'', \textit{et cetera}, you're good to go!

\section{What is GNU}

As we are talking about GNU Bash, it makes sense to quickly explain what GNU is. GNU stands for GNU's Not Unix, GNU tools are the tools that came with the original Unix operating system, rewritten and released under the GNU General Public License.\\

\href{https://www.gnu.org/}{The GNU Project}

\begin{itemize}
\item Freedom to use for any purpose
\item Freedom to study the software
\item Freedom to modify
\item Freedom to copy and share the software with others
\end{itemize}

\section{Interpreting the Prompt}

When you see something like ``\verb|user@host:~$|'' or ``\verb|[user@host ~] $|'', the shell is waiting for you to type a command.

The default prompt on most systems consists of four parts:
\begin{enumerate}
\item Your username;
\item Your computer's hostname;
\item Your current working directory;
\item An actual prompt, which is ``\verb|$|'' for normal users and ``\verb|#|'' for root, the superuser.
\end{enumerate}

\danger{The ``\verb|#|'' prompt actually signifies a UID of \verb|0|. If you have a setup where a user that is not ``\verb|root|'' has a UID of \verb|0| (and in this case such a user also has escalated privileges, because it's the UID rather than the username that matters, though most programs are not designed to work like this and can cause issues), its default prompt would also be ``\verb|#|''.}

Just know that such a prompt means that ``the last command (if any) has finished and the shell is waiting for a command''.

\section{Reading the Manual}

The system manual is a good resource for almost any command. To access the system manual, you may use the \verb|man| command.

Now try it!  Type \verb|man| and press the Enter/Return key on your keyboard.
\begin{verbatim}
user@host:~$ man
What manual page do you want?
For example, try 'man man'.
user@host:~$
\end{verbatim}

It looks like that the computer is telling you that \verb|man| doesn't know what manual page to look up \ldots You should think of some keyword to tell \verb|man| to look up as \verb|man| allows you to view specific pages in the system manual. Usually, the entries' names are the name of the program, system call names, standard library function names, file formats (specifically filenames), etc. They are divided into a few sections, which you should read more about with \verb|man man|, i.e. asking the system manual about the manual utility itself. It is wise use the system manual for the command in interest before asking for support---because many times the manual solves your problems, and people don't like repeating a documented solution all the time, hence the ``Read the F* Manual'' ``meme'' spreading around.

When you encounter something that you want to do but you're not sure which command to use, the \verb|apropos| command is useful. \verb|apropos| searches the manual database's page names and summary lines for your search term. For example, if you're looking for a utility to print a file (which is usually \verb|lpr| or \verb|lp|, by the way), you could run \verb|apropos print|, and you'll be presented with a list of related manual entries (though do note how the term ``print'' is used for ``output text onto the screen'' besides ``tell the printer to put ink (or powder) onto paper'' for historical reasons). Since you're looking for a \emph{command} to print, you'd generally want to look in section~1 of the manual (those marked with ``\verb|(1)|'').

Sometimes, you are not looking for the command of an external program, but rather a shell builtin, such as ``\verb|cd|''. In this case, you should use the ``\verb|help|'' command instead of ``\verb|man|''.

\danger{You may have encountered the ``\verb|info|'' command as a system manual command. That is part of the GNU \TeX info manual system, not the traditional UNIX manual system. If you are looking for the manual of GNU utilities, such as ``\verb|bash|'', the \TeX info manuals will be more comprehensive than the manpages, and will generally include more examples and friendly information than what manpages will tell you. However, not many programs outside of GNU utilities have \TeX info manuals.}

\section{Useful general commands}
``\verb|clear|'' clears the screen
``\verb|whoami|'' displays current user name
``\verb|id|''  display user information
``\verb|logout|'' logout of current session
``\verb|date|'' display current date and time
``\verb|ln|'' make links between files

\section{File Operations}

Your UNIX-like operating system has a directory structure, and files are the centre of UNIX-like operating systems. Therefore, learning how to navigate the filesystem and handle files is important.

\subsection{Working Directory}

In any process, there is a \textit{current working directory}. Any filenames that do not begin with a ``\verb|/|'', i.e. those that are not absolute paths, are relative to the process's current working directory. At your shell prompt, you can use the ``\verb|pwd|'' ( present working directory ) command to know your current working directory. This might not seem useful at first as your shell prompt already lists your current working directory, but in the future it will be useful in scripts, where you want to do something like ``save the current location to a variable and return to it sometime later''.

You would see that your shell prompt says that ``\verb|~|'' is your working directory, while ``\verb|pwd|'' says something like ``\verb|/home/user|''. That's because ``\verb|~|'' is an abbreviation to your user's \textit{home directory} (``\verb|$HOME|''), which is usually where you land when you login and is the place a user stores their personal files.

**

this section needs a creative commons diagram of the GNU / Linux directory structure.

**

\subsection{Listing Files}

To list the files and directories of a given directory, use ``\texttt{ls \ph{directory}}''. This lists all the items inside \ph{directory}. If \ph{directory} is unspecified, it lists the current working directory by default.

By default, \verb|ls| does not list hidden files, i.e. files whose filename beings with a dot (``\verb|.|''). If you pass the ``\verb|-a|`` option to it, hidden files are now listed. Note that (``\verb|.|'') and (``\verb|..|'') as independent filenames mean ``current directory'' and ``parent directory'' respectively.

By default, \verb|ls| only lists the filenames of the files. You may want more information. To get more information, you may pass the ``\verb|-l|'' option, which lets \verb|ls| display extra information like filesystem permissions, file sizes (default unit is bytes, use ``\verb|-h|'' for adaptive units), modification dates, \textit{et cetera}.

\subsection{Reading Files}

You may use the ``\verb|cat|'' command to output the data in a specified filename onto ``\verb|cat|'''s standard output (usually the terminal). If multiple filenames are specified as arguments, the files are concatenated.

While the ``\verb|cat|'' command can be used to read files, if the file is long then the output of cat will just scroll the contents up the screen. The ``\verb|less|'' command displays the file content, but allows user interaction to display the file a page at a time.

If you are using ``\verb|less|'' you can press ``\verb|q|'' to quit and return to the bash prompt.

You can also use the ``\verb|more|'' command to view a file a page at a time, as with less, press q to quit.

The `\verb|diff|'' command can be used to display any text that is different between two files. This is useful if you can't remember which is a later version for example.

You can also use the ``\verb|grep|'' to search through a file for specific text,
``\verb|grep pear fruit.txt|'' will search for the text pear in the file fruit.txt.

``\verb|grep|'' stands for GNU Regular Expression Parser and is very powerful so may be discussed later in a video.

The ``\verb|wc|'' command stands for word count, and as you would expect gives the number of words in a document.

\begin{verbatim}
~/test$ wc fruit.txt
 5  5 31 fruit.txt
\end{verbatim}

In this case, 5 lines, 5 words and 31 characters.



\subsection{Creating Files}

To create a file, you can use the ``\verb|touch|'', in this case ``\verb|touch file.txt|'' will create an empty file called file.txt.

``\verb|touch|'' will also allow you to create multiple files, for example ``\verb|touch file.txt file2.txt file3.txt|''

To work with longer files, you can use editors such as ``\verb|nano|'' can also be used so ``\verb|nano file.txt|'' will either open an existing file or create a new file with that file name. You can then edit the file contents.

You can also create files by directing the output of one command in to another. For example, we have previously discussed the ``\verb|ls|'' command, so using ``\verb|ls > filelist.txt|'' will direct the output of ``\verb|ls|'' in to a file called ``\verb|filelist.txt|''. Note that this truncates (i.e. clear) filelist.txt before redirecting the stream. If you want to append instead, use ``\verb|>>|'' instead of ``\verb|>|''.

\danger{It is a bad idea to parse the output of \verb|ls| in scripts. \verb|ls| uses special output formats (dollar symbol notations) and has other caveats for strange filenames. Evaluating the output is potentially a security risk, too. Batch operations should either use shell wildcard or the ``\verb|find|'' command for better robustness.}

\subsection{File manipulation}

Bash allows you to move, copy and rename files.

``\verb|cp|'' copy a file
``\verb|mv|'' move a file or this will also rename a file
``\verb|rename|'' This allows you to rename a file or set of files. Very useful for batch renaming for example:-
``\verb|rename 's/DSCF/newname/' *|'' will take a set of files in this case starting with DSCF and replace that with newname. If you have lots of files on your camera DSCF is an example of a prefix at the start of each photo filename. If you go to the beach you may want to rename that to beach for example, so that set of photos can be identified later. It just renames that portion of the file, so the numerical part remains intact.
``\verb|rm|'' remove or delete a file
``\verb|mkdir|'' create a directory
``\verb|rmdir|'' remove or delete a directory

\subsection{File Permissions}

File permissions are an important security feature in any Unix type environment. Files can either be Read, Write or Executable (e.g a shell script). Depending on who you want to be able to access these files, a file can either be just for the user who created it, a group of users, or anyone else.

The two main commands that are important here are

``\verb|chmod|'' - Changes the file permission, to determine what can be done with it.
``\verb|chown|'' - Changes the file owner
``\verb|chgrp|'' - Changes group owner

\begin{table}[h]
\begin{tabular}{|ccc|ccc|ccc|}
\hline
\multicolumn{3}{|c|}{\textbf{USER}}                               & \multicolumn{3}{c|}{\textbf{GROUP}}                              & \multicolumn{3}{c|}{\textbf{OTHER}}                              \\ \hline
\multicolumn{1}{|c|}{Read} & \multicolumn{1}{c|}{Write} & Execute & \multicolumn{1}{c|}{Read} & \multicolumn{1}{c|}{Write} & Execute & \multicolumn{1}{c|}{Read} & \multicolumn{1}{c|}{Write} & Execute \\ \hline
\multicolumn{1}{|c|}{400}  & \multicolumn{1}{c|}{200}   & 100     & \multicolumn{1}{c|}{40}   & \multicolumn{1}{c|}{20}    & 10      & \multicolumn{1}{c|}{4}    & \multicolumn{1}{c|}{2}     & 1       \\ \hline
\end{tabular}
\end{table}


\subsection{Finding Documents and Files}

The ``\verb|find|'' command will allow you to locate a file or set of files on you system. Basic usage is ``\verb|find -n "file.txt" |''

If you want to find where a particular command is on your system you can use ``\verb|which|''. So an example of this could be ``\verb|which bash|'' the command will then return ``\verb|/usr/bin/bash|''

\subsection{User accounts and passwords}

It is sometimes necessary to add a new user to your system, or remove a user, you can also change a users password if they forget what it is.

\begin{itemize}
\item ``\verb|adduser|'' - add a new user to the system
\item ``\verb|useradd|'' - add a user to a group
\item ``\verb|deluser|'' - delete a user from the system
\item ``\verb|groupdel / delgroup|'' - manage groups on the system
\item ``\verb|addgroup|'' - add group to the system
\item ``\verb|passwd|'' - changes a user's password
\end{itemize}

use of the ``\verb|passwd|'' command usually combines with a user name

``\verb|passwd|'' changes currently logged in users password

\begin{verbatim}
$ passwd
Changing password for user1.
Current password:
New password:
Retype new password:
passwd: password updated successfully
\end{verbatim}

As previously mentioned the ``\verb|man|'' command can be used to find out more information about each of these.

\subsection{Managing Software}

Depending on what distribution you are using, there are several tools that can be used to install or remove software on a GNU / Linux system.

\begin{itemize}
\item ``\verb|apt|'' - Used on Debian and Debian derived systems e.g Ubuntu, Mint
\item ``\verb|rpm|'' - Used on Red Hat, and Red Hat derived systems e.g Fedora, SUSE
\item ``\verb|yum|'' - Frontend for ``\verb|rpm|''
\item ``\verb|dnf|'' - Newer frontend for ``\verb|rpm|'', used on Fedora and Red Hat
\item ``\verb|zypper|'' - Frontend for ``\verb|rpm|'', used on SUSE
\item ``\verb|guix|'' - Used on GNU Guix
\end{itemize}

\subsection{System Administration}

System administration is important. It helps you monitor users, monitor resources etc.
\begin{itemize}
\item[``\texttt{ps}''] lists running processes on your system
\item[``\texttt{top}''] displays processes and allows them to be killed, or change their process priority
\item[``\texttt{nice}'' \& ``\texttt{renice}''] - change process priority
\item[``\texttt{htop}''] more advanced version of top
\item[``\texttt{killall}''] kills a process e.g killall firefox-esr
\item[``\texttt{cron}''] Run a process at a specified time
\item[``\texttt{su}''] Drop to super user - Maintain current user \verb|$PATH|
\item[``\texttt{su-l }''] Drop to super user, using the root user's \verb|$PATH|
\item[``\texttt{sudo}''] super user do, drop to root to just run one command.
\item[``\texttt{reboot}''] restarts the system
\end{itemize}


\subsection{The Path Environment}
 In the previous subsection I mentioned the ``\verb|su|'' and ``\verb|su -|'' lets have a quick look at how the path environment compares, we can view the current path with ``\verb|echo $PATH|''
\begin{itemize}
\item This is the user path /usr/local/bin:/usr/bin:/bin:/usr/local/games:/usr/games
\item This is the root path when using ``\verb|su|'' /usr/local/bin:/usr/bin:/bin:/usr/local/games:/usr/games
\item This is the root path when using ``\verb|su -|'' /usr/local/sbin:/usr/local/bin:/usr/sbin:/usr/bin:/sbin:/bin:/snap/bin
\item
\end{itemize}

The difference is mostly that using ``\verb|su -|'' you won't get file not found when trying to run a command that is reserved for the root user.

\subsection{Environment variables}

``\verb|echo $HOSTNAME|'' - Display hostname of system
``\verb|echo $USER|'' - Current user
``\verb|echo $PATH|'' - Display current users path setting

\subsection{Hardware related commands}

On a Unix or GNU / Linux system, everything is a file, this includes hardware, which is represented by a file. All these files are stored in the /dev folder.

``\verb|ls /dev|''

``\verb|lsusb|'' - lists USB devices
``\verb|lspci|'' - lists PCI devices
``\verb|lsblk|'' - lists block devices (e.g hard disks and removable media)

These commands are useful to figure out what is on your system, and where it is. Removable media can be mounted on to a folder, with a command such as ``\verb|mount /dev/usb /media/flash |''  then when you run ``\verb|cd /media/flash|'' you see the files on the device.

It can be a little more involved than this from the command shell though.

The ``\verb|lsblk|'' is very useful for writing USB flash disks, to create a install disk or writing to a SD card in order to create Raspberry Pi disk.

While not fully related to this the ``\verb|dmesg|'' command displays kernel messages, so you can run prior to, and after plugging in a device to see if it has been detected.

\subsection{Restarting an unresponsive system}

This may also be useful - \url{http://tuxdiary.com/2012/06/05/reiub-reset-unusable-linux-box/}

\



\subsection{Videos}

John Collins from \href{https://www.ezeelinux.com/}{EzeeLinux} produced an excellent series of videos covering different topics relating to the BASH shell, We have only touched on this above.

These are hosted on YouTube. For the purpose of this series, the links use the privacy friendly front end on Invidious:

\begin{itemize}
\item \youtube{eH8Z9zeywq0}{Access and Navigation}
\item \youtube{eH8Z9zeywq0}{Editing Text Files}
\item \youtube{s23NqWKxOXk}{Privileges and Permissions}
\item \youtube{4r7V2-EBnR0}{Finding Documentation and Files}
\item \youtube{XVCf0cou6EU}{User Accounts and Passwords}
\item \youtube{lNyJllHk2SA}{Managing Software}
\item \youtube{4\_21KZ3qKEI7}{System Administration Tools}
\item \youtube{57sp8Y0GL40}{Bash Scripting}
\end{itemize}


\section{Editing files}

The nano editor is installed on pretty much every GNU / Linux system, therefore it does make a good starting point, if we want to keep things simple. So before moving on to shell scripting, we should learn how to use this, or at least, open, save and edit files.

\begin{itemize}
\item \href{https://www.nano-editor.org/}{Nano Editor website}
\item \youtube{cLyUZAabf40}{Beginners' guide to nano}
\item \href{https://www.nano-editor.org/dist/latest/cheatsheet.html}{Nano Cheat sheet}

\end{itemize}

\section{Shell Scripting}

Once you have become more used to the command line interface, a shell script can make running the more labour intensive tasks much easier. A shell script is a list of commands within a text file, which will can be run manually or perhaps run at a set time on your system.

\subsection{Getting Started}


\subsection{Checking who you are}

Depending on the script being run and its function, you may want to check the user is running the script as root.

\begin{verbatim}
if [ "$EUID" -ne 0 ]; then
  echo "Please run as root / sudo"
  exit
fi
\end{verbatim}

Shell scripts start with ``\verb|#!|'' followed by a path to the interpreter, for example ``\verb|/bin/sh|''.
There is no standard location for where the interpreter is located, but ``\verb|/usr/bin/env|'' often (but not always) exists.
It is possible to use ``\verb|env|'' to find an interpreter without specifying the full path, for example:

\begin{verbatim}
#!/usr/bin/env bash
\end{verbatim}

\chapter{Next Steps}

\section{Networking}

It is a good idea to understand networking concepts, this is especially useful if you want to host your own web services.

\youtube{\_IOZ8\_cPgu8}{Networking Beginners}

\begin{itemize}
\item a switch
\item router
\item gateway
\item subnet
\item gateway
\item firewall
\item DMZ
\end{itemize}

As this is a series on bash, then the above will give you an introduction to networking. However within bash there are several tools for networking.

``\verb|ping|'' sends a packet of data to the specified remote computer, this helps to check the remote computer is responding.

``\verb|ping 127.0.0.1|'' will therefore ping the computer you are on, as \texttt{127.0.0.1} represents \texttt{localhost}. This will keep going until stopped with ``\verb|ctrl-d|''. You can ping other computers on your network or on the wider internet. Please \emph{do not flood} other computers.

``\verb|ip addr show|''

Shows local network information for each of the systems network interfaces

\begin{itemize}
\item ``\verb|hostname|'' will display the current system hostname
\item ``\verb|hostname -i|'' will display the localhost address
\item ``\verb|hostname -I|'' will display system IPv4 and IPv6 address
\end{itemize}

see ``\verb|man hostname|'' for full details

\begin{verbatim}

random text to fill in for now

\end{verbatim}



\section{OpenSSH}

OpenSSH is a part of a suite of tools that allow secure remote access to systems. Open Secure Shell.

\href{https://www.openssh.com/}{OpenSSH server}

Install with ``\verb|apt install openssh-server|''

SSH can be used with ``\verb|ssh-keygen|'' to generate a ssh keypair, which can be used to login to a remote system without a password. This is a good option to maintain security.

To generate a key pair with ``\verb|ssh-keygen|''


\begin{verbatim}

testuser@Desktop:~$ ssh-keygen
Generating public/private rsa key pair.
Enter file in which to save the key (/home/testuser/.ssh/id_rsa):
Created directory '/home/testuser/.ssh'.
Enter passphrase (empty for no passphrase):
Enter same passphrase again:
Your identification has been saved in /home/testuser/.ssh/id_rsa
Your public key has been saved in /home/testuser/.ssh/id_rsa.pub
The key fingerprint is:
SHA256:10eg+9WU5kaiaicN4oLP+0Y2L/FmCwKygIQjwo9mNq4 testuser@Desktop
The key's randomart image is:
+---[RSA 3072]----+
|            .    |
|o          . .  .|
|=o        .  ..+.|
|=.o        o..=o |
|oB o   .S.o.. oo.|
|* + o .=..+. o.  |
| o . ooo=+ o.    |
|.   o ooo=o      |
|E    ++.+..      |
+----[SHA256]-----+

\end{verbatim}

I have just accepted the defaults by pressing enter. for both the file name and setting and confirming a passphrase.

If you now type ``\verb|cd .ssh|'' you will find two files.

\begin{verbatim}
id_rsa  id_rsa.pub
\end{verbatim}

\verb|id_rsa| is your \textbf{Private} key and should be treated as such. \verb|id_rsa.pub| is your \textbf{Public} key and can be shared.

It is possible to specify the type of key to be generated with the ``\verb|-t|'' option, which can be used like:

\begin{verbatim}
ssh-keygen -t ed25519
\end{verbatim}

The default is ``\verb|rsa|'', but ``\verb|ed25519|'' is a more modern and faster, but still just as secure, algorithm.

\section{Git}

Git is a tool to help manage software projects, git is used in combination with sites such as \href{https://github.com/}{GitHub}, however there are several alternatives such as
\href{https://codeberg.org/}{Codeberg}, \href{https://sr.ht/}{Sourcehut} and others.

Git is clearly a very powerful tool, learning git is an important skill, but it can be complex and confusing at times, so it is really useful to be able to ask for help.



\section{Self Hosting}

Andrew Yu has produced a really good guide to self hosting, this can be found at

\href{https://host.andrewyu.org/}{Andrew Yu Hosting guide}

This covers :

\begin{itemize}
\item Get a domain name
\item Get a server
\item Set up DNS settings to connect your server and domain name
\item Set up your web server
\item Get a secure HTTPS connection with Certbot
\end{itemize}

\href{https://networkverge.com/common-ports/}{Common network ports}

There are also other guides to set up other types of server, or at least links to.

\section{Gaming}

Gaming may not be something that seems directly related to this. You may be interested in running a game server, for friends or to take part in events such as:

\href{https://libregaming.org/}{Libre gaming night}

There is a related IRC Channel on Libera Chat -- \libera{libregamenight}

\appendix
\chapter*{\rlap{GNU Free Documentation License}}
\phantomsection  % so hyperref creates bookmarks
\addcontentsline{toc}{chapter}{GNU Free Documentation License}
\label{label_fdl}

 \begin{center}

       Version 1.3, 3 November 2008


 Copyright \copyright{} 2000, 2001, 2002, 2007, 2008  Free Software Foundation, Inc.
 
 \bigskip
 
     \texttt{<https://fsf.org/>}
  
 \bigskip
 
 Everyone is permitted to copy and distribute verbatim copies
 of this license document, but changing it is not allowed.
\end{center}


\begin{center}
{\bf\large Preamble}
\end{center}

The purpose of this License is to make a manual, textbook, or other
functional and useful document ``free'' in the sense of freedom: to
assure everyone the effective freedom to copy and redistribute it,
with or without modifying it, either commercially or noncommercially.
Secondarily, this License preserves for the author and publisher a way
to get credit for their work, while not being considered responsible
for modifications made by others.

This License is a kind of ``copyleft'', which means that derivative
works of the document must themselves be free in the same sense.  It
complements the GNU General Public License, which is a copyleft
license designed for free software.

We have designed this License in order to use it for manuals for free
software, because free software needs free documentation: a free
program should come with manuals providing the same freedoms that the
software does.  But this License is not limited to software manuals;
it can be used for any textual work, regardless of subject matter or
whether it is published as a printed book.  We recommend this License
principally for works whose purpose is instruction or reference.


\begin{center}
{\Large\bf 1. APPLICABILITY AND DEFINITIONS\par}
\phantomsection
\addcontentsline{toc}{section}{1. APPLICABILITY AND DEFINITIONS}
\end{center}

This License applies to any manual or other work, in any medium, that
contains a notice placed by the copyright holder saying it can be
distributed under the terms of this License.  Such a notice grants a
world-wide, royalty-free license, unlimited in duration, to use that
work under the conditions stated herein.  The ``\textbf{Document}'', below,
refers to any such manual or work.  Any member of the public is a
licensee, and is addressed as ``\textbf{you}''.  You accept the license if you
copy, modify or distribute the work in a way requiring permission
under copyright law.

A ``\textbf{Modified Version}'' of the Document means any work containing the
Document or a portion of it, either copied verbatim, or with
modifications and/or translated into another language.

A ``\textbf{Secondary Section}'' is a named appendix or a front-matter section of
the Document that deals exclusively with the relationship of the
publishers or authors of the Document to the Document's overall subject
(or to related matters) and contains nothing that could fall directly
within that overall subject.  (Thus, if the Document is in part a
textbook of mathematics, a Secondary Section may not explain any
mathematics.)  The relationship could be a matter of historical
connection with the subject or with related matters, or of legal,
commercial, philosophical, ethical or political position regarding
them.

The ``\textbf{Invariant Sections}'' are certain Secondary Sections whose titles
are designated, as being those of Invariant Sections, in the notice
that says that the Document is released under this License.  If a
section does not fit the above definition of Secondary then it is not
allowed to be designated as Invariant.  The Document may contain zero
Invariant Sections.  If the Document does not identify any Invariant
Sections then there are none.

The ``\textbf{Cover Texts}'' are certain short passages of text that are listed,
as Front-Cover Texts or Back-Cover Texts, in the notice that says that
the Document is released under this License.  A Front-Cover Text may
be at most 5 words, and a Back-Cover Text may be at most 25 words.

A ``\textbf{Transparent}'' copy of the Document means a machine-readable copy,
represented in a format whose specification is available to the
general public, that is suitable for revising the document
straightforwardly with generic text editors or (for images composed of
pixels) generic paint programs or (for drawings) some widely available
drawing editor, and that is suitable for input to text formatters or
for automatic translation to a variety of formats suitable for input
to text formatters.  A copy made in an otherwise Transparent file
format whose markup, or absence of markup, has been arranged to thwart
or discourage subsequent modification by readers is not Transparent.
An image format is not Transparent if used for any substantial amount
of text.  A copy that is not ``Transparent'' is called ``\textbf{Opaque}''.

Examples of suitable formats for Transparent copies include plain
ASCII without markup, Texinfo input format, LaTeX input format, SGML
or XML using a publicly available DTD, and standard-conforming simple
HTML, PostScript or PDF designed for human modification.  Examples of
transparent image formats include PNG, XCF and JPG.  Opaque formats
include proprietary formats that can be read and edited only by
proprietary word processors, SGML or XML for which the DTD and/or
processing tools are not generally available, and the
machine-generated HTML, PostScript or PDF produced by some word
processors for output purposes only.

The ``\textbf{Title Page}'' means, for a printed book, the title page itself,
plus such following pages as are needed to hold, legibly, the material
this License requires to appear in the title page.  For works in
formats which do not have any title page as such, ``Title Page'' means
the text near the most prominent appearance of the work's title,
preceding the beginning of the body of the text.

The ``\textbf{publisher}'' means any person or entity that distributes
copies of the Document to the public.

A section ``\textbf{Entitled XYZ}'' means a named subunit of the Document whose
title either is precisely XYZ or contains XYZ in parentheses following
text that translates XYZ in another language.  (Here XYZ stands for a
specific section name mentioned below, such as ``\textbf{Acknowledgements}'',
``\textbf{Dedications}'', ``\textbf{Endorsements}'', or ``\textbf{History}''.)  
To ``\textbf{Preserve the Title}''
of such a section when you modify the Document means that it remains a
section ``Entitled XYZ'' according to this definition.

The Document may include Warranty Disclaimers next to the notice which
states that this License applies to the Document.  These Warranty
Disclaimers are considered to be included by reference in this
License, but only as regards disclaiming warranties: any other
implication that these Warranty Disclaimers may have is void and has
no effect on the meaning of this License.


\begin{center}
{\Large\bf 2. VERBATIM COPYING\par}
\phantomsection
\addcontentsline{toc}{section}{2. VERBATIM COPYING}
\end{center}

You may copy and distribute the Document in any medium, either
commercially or noncommercially, provided that this License, the
copyright notices, and the license notice saying this License applies
to the Document are reproduced in all copies, and that you add no other
conditions whatsoever to those of this License.  You may not use
technical measures to obstruct or control the reading or further
copying of the copies you make or distribute.  However, you may accept
compensation in exchange for copies.  If you distribute a large enough
number of copies you must also follow the conditions in section~3.

You may also lend copies, under the same conditions stated above, and
you may publicly display copies.


\begin{center}
{\Large\bf 3. COPYING IN QUANTITY\par}
\phantomsection
\addcontentsline{toc}{section}{3. COPYING IN QUANTITY}
\end{center}


If you publish printed copies (or copies in media that commonly have
printed covers) of the Document, numbering more than 100, and the
Document's license notice requires Cover Texts, you must enclose the
copies in covers that carry, clearly and legibly, all these Cover
Texts: Front-Cover Texts on the front cover, and Back-Cover Texts on
the back cover.  Both covers must also clearly and legibly identify
you as the publisher of these copies.  The front cover must present
the full title with all words of the title equally prominent and
visible.  You may add other material on the covers in addition.
Copying with changes limited to the covers, as long as they preserve
the title of the Document and satisfy these conditions, can be treated
as verbatim copying in other respects.

If the required texts for either cover are too voluminous to fit
legibly, you should put the first ones listed (as many as fit
reasonably) on the actual cover, and continue the rest onto adjacent
pages.

If you publish or distribute Opaque copies of the Document numbering
more than 100, you must either include a machine-readable Transparent
copy along with each Opaque copy, or state in or with each Opaque copy
a computer-network location from which the general network-using
public has access to download using public-standard network protocols
a complete Transparent copy of the Document, free of added material.
If you use the latter option, you must take reasonably prudent steps,
when you begin distribution of Opaque copies in quantity, to ensure
that this Transparent copy will remain thus accessible at the stated
location until at least one year after the last time you distribute an
Opaque copy (directly or through your agents or retailers) of that
edition to the public.

It is requested, but not required, that you contact the authors of the
Document well before redistributing any large number of copies, to give
them a chance to provide you with an updated version of the Document.


\begin{center}
{\Large\bf 4. MODIFICATIONS\par}
\phantomsection
\addcontentsline{toc}{section}{4. MODIFICATIONS}
\end{center}

You may copy and distribute a Modified Version of the Document under
the conditions of sections 2 and 3 above, provided that you release
the Modified Version under precisely this License, with the Modified
Version filling the role of the Document, thus licensing distribution
and modification of the Modified Version to whoever possesses a copy
of it.  In addition, you must do these things in the Modified Version:

\begin{itemize}
\item[A.] 
   Use in the Title Page (and on the covers, if any) a title distinct
   from that of the Document, and from those of previous versions
   (which should, if there were any, be listed in the History section
   of the Document).  You may use the same title as a previous version
   if the original publisher of that version gives permission.
   
\item[B.]
   List on the Title Page, as authors, one or more persons or entities
   responsible for authorship of the modifications in the Modified
   Version, together with at least five of the principal authors of the
   Document (all of its principal authors, if it has fewer than five),
   unless they release you from this requirement.
   
\item[C.]
   State on the Title page the name of the publisher of the
   Modified Version, as the publisher.
   
\item[D.]
   Preserve all the copyright notices of the Document.
   
\item[E.]
   Add an appropriate copyright notice for your modifications
   adjacent to the other copyright notices.
   
\item[F.]
   Include, immediately after the copyright notices, a license notice
   giving the public permission to use the Modified Version under the
   terms of this License, in the form shown in the Addendum below.
   
\item[G.]
   Preserve in that license notice the full lists of Invariant Sections
   and required Cover Texts given in the Document's license notice.
   
\item[H.]
   Include an unaltered copy of this License.
   
\item[I.]
   Preserve the section Entitled ``History'', Preserve its Title, and add
   to it an item stating at least the title, year, new authors, and
   publisher of the Modified Version as given on the Title Page.  If
   there is no section Entitled ``History'' in the Document, create one
   stating the title, year, authors, and publisher of the Document as
   given on its Title Page, then add an item describing the Modified
   Version as stated in the previous sentence.
   
\item[J.]
   Preserve the network location, if any, given in the Document for
   public access to a Transparent copy of the Document, and likewise
   the network locations given in the Document for previous versions
   it was based on.  These may be placed in the ``History'' section.
   You may omit a network location for a work that was published at
   least four years before the Document itself, or if the original
   publisher of the version it refers to gives permission.
   
\item[K.]
   For any section Entitled ``Acknowledgements'' or ``Dedications'',
   Preserve the Title of the section, and preserve in the section all
   the substance and tone of each of the contributor acknowledgements
   and/or dedications given therein.
   
\item[L.]
   Preserve all the Invariant Sections of the Document,
   unaltered in their text and in their titles.  Section numbers
   or the equivalent are not considered part of the section titles.
   
\item[M.]
   Delete any section Entitled ``Endorsements''.  Such a section
   may not be included in the Modified Version.
   
\item[N.]
   Do not retitle any existing section to be Entitled ``Endorsements''
   or to conflict in title with any Invariant Section.
   
\item[O.]
   Preserve any Warranty Disclaimers.
\end{itemize}

If the Modified Version includes new front-matter sections or
appendices that qualify as Secondary Sections and contain no material
copied from the Document, you may at your option designate some or all
of these sections as invariant.  To do this, add their titles to the
list of Invariant Sections in the Modified Version's license notice.
These titles must be distinct from any other section titles.

You may add a section Entitled ``Endorsements'', provided it contains
nothing but endorsements of your Modified Version by various
parties---for example, statements of peer review or that the text has
been approved by an organization as the authoritative definition of a
standard.

You may add a passage of up to five words as a Front-Cover Text, and a
passage of up to 25 words as a Back-Cover Text, to the end of the list
of Cover Texts in the Modified Version.  Only one passage of
Front-Cover Text and one of Back-Cover Text may be added by (or
through arrangements made by) any one entity.  If the Document already
includes a cover text for the same cover, previously added by you or
by arrangement made by the same entity you are acting on behalf of,
you may not add another; but you may replace the old one, on explicit
permission from the previous publisher that added the old one.

The author(s) and publisher(s) of the Document do not by this License
give permission to use their names for publicity for or to assert or
imply endorsement of any Modified Version.


\begin{center}
{\Large\bf 5. COMBINING DOCUMENTS\par}
\phantomsection
\addcontentsline{toc}{section}{5. COMBINING DOCUMENTS}
\end{center}


You may combine the Document with other documents released under this
License, under the terms defined in section~4 above for modified
versions, provided that you include in the combination all of the
Invariant Sections of all of the original documents, unmodified, and
list them all as Invariant Sections of your combined work in its
license notice, and that you preserve all their Warranty Disclaimers.

The combined work need only contain one copy of this License, and
multiple identical Invariant Sections may be replaced with a single
copy.  If there are multiple Invariant Sections with the same name but
different contents, make the title of each such section unique by
adding at the end of it, in parentheses, the name of the original
author or publisher of that section if known, or else a unique number.
Make the same adjustment to the section titles in the list of
Invariant Sections in the license notice of the combined work.

In the combination, you must combine any sections Entitled ``History''
in the various original documents, forming one section Entitled
``History''; likewise combine any sections Entitled ``Acknowledgements'',
and any sections Entitled ``Dedications''.  You must delete all sections
Entitled ``Endorsements''.

\begin{center}
{\Large\bf 6. COLLECTIONS OF DOCUMENTS\par}
\phantomsection
\addcontentsline{toc}{section}{6. COLLECTIONS OF DOCUMENTS}
\end{center}

You may make a collection consisting of the Document and other documents
released under this License, and replace the individual copies of this
License in the various documents with a single copy that is included in
the collection, provided that you follow the rules of this License for
verbatim copying of each of the documents in all other respects.

You may extract a single document from such a collection, and distribute
it individually under this License, provided you insert a copy of this
License into the extracted document, and follow this License in all
other respects regarding verbatim copying of that document.


\begin{center}
{\Large\bf 7. AGGREGATION WITH INDEPENDENT WORKS\par}
\phantomsection
\addcontentsline{toc}{section}{7. AGGREGATION WITH INDEPENDENT WORKS}
\end{center}


A compilation of the Document or its derivatives with other separate
and independent documents or works, in or on a volume of a storage or
distribution medium, is called an ``aggregate'' if the copyright
resulting from the compilation is not used to limit the legal rights
of the compilation's users beyond what the individual works permit.
When the Document is included in an aggregate, this License does not
apply to the other works in the aggregate which are not themselves
derivative works of the Document.

If the Cover Text requirement of section~3 is applicable to these
copies of the Document, then if the Document is less than one half of
the entire aggregate, the Document's Cover Texts may be placed on
covers that bracket the Document within the aggregate, or the
electronic equivalent of covers if the Document is in electronic form.
Otherwise they must appear on printed covers that bracket the whole
aggregate.


\begin{center}
{\Large\bf 8. TRANSLATION\par}
\phantomsection
\addcontentsline{toc}{section}{8. TRANSLATION}
\end{center}


Translation is considered a kind of modification, so you may
distribute translations of the Document under the terms of section~4.
Replacing Invariant Sections with translations requires special
permission from their copyright holders, but you may include
translations of some or all Invariant Sections in addition to the
original versions of these Invariant Sections.  You may include a
translation of this License, and all the license notices in the
Document, and any Warranty Disclaimers, provided that you also include
the original English version of this License and the original versions
of those notices and disclaimers.  In case of a disagreement between
the translation and the original version of this License or a notice
or disclaimer, the original version will prevail.

If a section in the Document is Entitled ``Acknowledgements'',
``Dedications'', or ``History'', the requirement (section~4) to Preserve
its Title (section~1) will typically require changing the actual
title.


\begin{center}
{\Large\bf 9. TERMINATION\par}
\phantomsection
\addcontentsline{toc}{section}{9. TERMINATION}
\end{center}


You may not copy, modify, sublicense, or distribute the Document
except as expressly provided under this License.  Any attempt
otherwise to copy, modify, sublicense, or distribute it is void, and
will automatically terminate your rights under this License.

However, if you cease all violation of this License, then your license
from a particular copyright holder is reinstated (a) provisionally,
unless and until the copyright holder explicitly and finally
terminates your license, and (b) permanently, if the copyright holder
fails to notify you of the violation by some reasonable means prior to
60 days after the cessation.

Moreover, your license from a particular copyright holder is
reinstated permanently if the copyright holder notifies you of the
violation by some reasonable means, this is the first time you have
received notice of violation of this License (for any work) from that
copyright holder, and you cure the violation prior to 30 days after
your receipt of the notice.

Termination of your rights under this section does not terminate the
licenses of parties who have received copies or rights from you under
this License.  If your rights have been terminated and not permanently
reinstated, receipt of a copy of some or all of the same material does
not give you any rights to use it.


\begin{center}
{\Large\bf 10. FUTURE REVISIONS OF THIS LICENSE\par}
\phantomsection
\addcontentsline{toc}{section}{10. FUTURE REVISIONS OF THIS LICENSE}
\end{center}


The Free Software Foundation may publish new, revised versions
of the GNU Free Documentation License from time to time.  Such new
versions will be similar in spirit to the present version, but may
differ in detail to address new problems or concerns.  See
\texttt{https://www.gnu.org/licenses/}.

Each version of the License is given a distinguishing version number.
If the Document specifies that a particular numbered version of this
License ``or any later version'' applies to it, you have the option of
following the terms and conditions either of that specified version or
of any later version that has been published (not as a draft) by the
Free Software Foundation.  If the Document does not specify a version
number of this License, you may choose any version ever published (not
as a draft) by the Free Software Foundation.  If the Document
specifies that a proxy can decide which future versions of this
License can be used, that proxy's public statement of acceptance of a
version permanently authorizes you to choose that version for the
Document.


\begin{center}
{\Large\bf 11. RELICENSING\par}
\phantomsection
\addcontentsline{toc}{section}{11. RELICENSING}
\end{center}


``Massive Multiauthor Collaboration Site'' (or ``MMC Site'') means any
World Wide Web server that publishes copyrightable works and also
provides prominent facilities for anybody to edit those works.  A
public wiki that anybody can edit is an example of such a server.  A
``Massive Multiauthor Collaboration'' (or ``MMC'') contained in the
site means any set of copyrightable works thus published on the MMC
site.

``CC-BY-SA'' means the Creative Commons Attribution-Share Alike 3.0
license published by Creative Commons Corporation, a not-for-profit
corporation with a principal place of business in San Francisco,
California, as well as future copyleft versions of that license
published by that same organization.

``Incorporate'' means to publish or republish a Document, in whole or
in part, as part of another Document.

An MMC is ``eligible for relicensing'' if it is licensed under this
License, and if all works that were first published under this License
somewhere other than this MMC, and subsequently incorporated in whole
or in part into the MMC, (1) had no cover texts or invariant sections,
and (2) were thus incorporated prior to November 1, 2008.

The operator of an MMC Site may republish an MMC contained in the site
under CC-BY-SA on the same site at any time before August 1, 2009,
provided the MMC is eligible for relicensing.


\begin{center}
{\Large\bf ADDENDUM: How to use this License for your documents\par}
\phantomsection
\addcontentsline{toc}{section}{ADDENDUM: How to use this License for your documents}
\end{center}

To use this License in a document you have written, include a copy of
the License in the document and put the following copyright and
license notices just after the title page:

\bigskip
\begin{quote}
    Copyright \copyright{}  YEAR  YOUR NAME.
    Permission is granted to copy, distribute and/or modify this document
    under the terms of the GNU Free Documentation License, Version 1.3
    or any later version published by the Free Software Foundation;
    with no Invariant Sections, no Front-Cover Texts, and no Back-Cover Texts.
    A copy of the license is included in the section entitled ``GNU
    Free Documentation License''.
\end{quote}
\bigskip
    
If you have Invariant Sections, Front-Cover Texts and Back-Cover Texts,
replace the ``with \dots\ Texts.''\ line with this:

\bigskip
\begin{quote}
    with the Invariant Sections being LIST THEIR TITLES, with the
    Front-Cover Texts being LIST, and with the Back-Cover Texts being LIST.
\end{quote}
\bigskip
    
If you have Invariant Sections without Cover Texts, or some other
combination of the three, merge those two alternatives to suit the
situation.

If your document contains nontrivial examples of program code, we
recommend releasing these examples in parallel under your choice of
free software license, such as the GNU General Public License,
to permit their use in free software.


\end{document}
